\chapter{Conclusion} \label{chapter:conclusion}
\thispagestyle{empty}% no page number in chapter title page

The increasing popularity of video telephony over mobile platforms like smartphones has increased the demand for efficient low bitrate solutions. The availability of high speed mobile internet has not kept its pace with the demand in many parts of the world. It is therefore important to develop efficient low bitrate solutions that enable high quality video telephony over slow cellular networks. 

This thesis contributes to the research area by proposing a ROI-based bitrate control for low delay encoding to increase the perceptual quality during video conferencing by preferentially encoding face regions (ROI) with higher quality compared to the rest of the frame. As a first step, an existing low delay bitrate control module's performance is evaluated at low bitrate using multiple sample video conferencing sequences. The H.264 encoder is used for all the evaluations in this work. This work proposes two different ways of achieving better perceptual quality at low bitrate. These approaches offer a trade-off between implementation complexity and output quality. 

Firstly, a simple approach of using negative QP offset for macroblocks belonging to ROI is studied. This approach requires very minimal changes to the existing low-delay bitrate control module. Further improvements like tuning QP offsets, using area of ROI for QP offset computation and usage of bi-direction QP offsets for both ROI and non-ROI are described. This approach of QP offset based ROI encoding offers considerable improvements in overall perceptual quality compared to conventional encoding. It is established through experiments with different magnitude of QP offsets that improving quality in ROI at the cost of degrading quality in non-ROI improves the perceptual quality with increase in magnitude of QP offset up to a certain point. The usage of a very high magnitude QP offset leading to an unbounded movement of bits from ROI to non-ROI can have an adverse effect on the perceptual quality.

The second approach discussed in this work includes ROI-based bit-allocation to allocate additional bits to ROI to improve the perceptual quality. In this approach, explicit region based bit-allocation is performed where a target bitrate is assigned to ROI and non-ROI parts independently. In order to make this approach more robust across different classes of input video contents, the characteristics of input frame complexity are considered to compute the bits allocated for ROI to achieve an improved perceptual quality. Therefore, the magnitude of bit movement from non-ROI to ROI parts is optimal for the specific content to be encoded. This is accomplished by developing a cost-bits model to predict bit-consumption using the complexity of a region (RDO cost). Finally, the output of conventional encoding is compared against the two proposed approaches for ROI-based encoding. The evaluation includes the PSNR values, PSNR and QP distribution within a frame. The delay curve for each of the encoded sequences is analyzed to make sure that ROI-based encoding does not alter the overall behavior of the encoder. It is found that the ROI-based bit-allocation offers superior perceptual quality output with well defined movement of bits from non-ROI to ROI compared to ROI-based encoding using QP offset. However, both the proposed approaches offer better perceptual quality compared to the conventional encoding.

\section{Future Work}
There is a lot of scope to improve the perceptual quality at low bitrate by removing the perceptual redundancies. This work proposes two approaches to achieve this in low-delay encoding. The results obtained in this work can be improved further by developing more accurate cost-bits model which considers the changes in QP within a frame due to varying complexities across macroblocks. 

This work is proposed on the fundamental assumption that improving the quality of the ROI at the cost of non-ROI improves the overall perceptual quality. The magnitude of this quality difference is decided based on the relative complexities of these regions. Therefore this work offers a way to remove the arbitrary weights to some extent for ROI in ROI based bit-allocation used in the previous works. However, the value of ROI bias factor chosen in this work is heuristic. There is no well defined formulation to determine the absolute magnitude of quality difference between ROI and non-ROI to achieve the best perceptual quality. This is an open research area which needs investigation in the fields like determination of subjective quality assessment metric.
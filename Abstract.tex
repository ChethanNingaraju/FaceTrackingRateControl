%\switchlanguage{en} % The abstract is supposed to be in English!

\thispagestyle{plain}

\section*{Abstract}
In video communication applications, coding artifacts are especially disturbing in the face region at low bitrates. In this work, region-of-interest (ROI) based encoding techniques are proposed to preferentially code the face region with a higher quality to improve the perceived visual quality of the video conferencing system. The face region is detected in the video stream and marked as ROI before encoding. 

The low-delay bitrate control proposed for H.264 video coding standard is extended to implement two different approaches of ROI-based encoding. In the first approach, QP offsets for ROI is computed based on the relative area of ROI to reduce the quantization parameter (QP) allocated for ROI macroblocks. In the second approach, region-based bit-allocation is performed to allocate a higher proportion of bits to ROI macroblocks. The two approaches offer a tradeoff between complexity and output quality. In contrast to the previous ROI-based video coding approaches, this work uses the complexities of ROI and non-ROI parts in a video frame to allocate an optimal amount of bits for the corresponding regions. Experimental results demonstrate that the quality of ROI is improved without any noticeable degradation in non-ROI quality hence improving the overall perceived visual quality at low bitrates.

In addition to ROI-based coding, optimization techniques are proposed for real-time face detection in a video stream by reusing motion vector information computed during motion estimation stage of video encoding. This work proposes motion vector based variable interval face detection technique instead of detecting the face in every frame of the video. The motion vectors of face regions are analyzed to perform face detection only when there is movement in the face region.

%\switchlanguage{\lang} % Switch back to the document's default language.

\chapter{Introduction}
\pagenumbering{arabic}%Ab hier, werden arabische Zahlen benutzt
\setcounter{page}{1}%Mit Abschnitt 1 beginnt die Seitennummerierung neu.
\thispagestyle{empty}

%Die Einleitung soll zum eigentlichen Themengebiet hinf�hren und die
%Motivation f�r die Arbeit liefern. Am Schlu� der Einleitung wird
%weiterhin noch eine �bersicht �ber die restliche Arbeit gegeben.
In recent years, there is increasing demand for high-quality video conferencing solutions. Due to availability of high-speed internet, video conferencing has proved to be an efficient alternative to face-to-face meetings. Video telephony has grown into a multi-billion dollar industry and has huge commercial significance. To address this growing need there has been constant improvement in  low-delay video coding techniques along with better techniques to ensure low-delay transmission reliability at the network level. The tremendous increase in smartphone usage has led to increase in video telephony over cellular networks whose bandwidth is highly constrained. Therefore, it is very important to develop methods of delivering high quality video with less bandwidth requirement. 

The most commonly used video coding standards like H.264/AVC have been designed to exploit the spatial and temporal redundancy in the input video stream to achieve high data compression. The techniques of spatial and temporal prediction form the core principle of these video coding standards. However, after encoding the video the perceptual redundancies still remain since human attention does not focus on the whole scene but only a small region of fixation called region-of-interest (ROI) \cite{Perception-model-of-face}. Therefore, reducing the perceptual redundancy gives a new dimension towards achieving lower bit-rate at acceptable perceptual quality.

The goal of this work is to identify the salient region of a frame, which is the face of the participant in a video conference. Since the attention of the viewer is mostly focused on the face of the other participants during a video conference call, improving the quality of the face region (ROI) can improve the overall perceptual quality. In this work, ideal capture conditions are assumed and results of the face tracking is used directly as additional information for the H264/AVC encoder's bitrate control. This work explores the methods of region of interest(ROI) based encoding to exploit the available bandwidth to encode regions that are of high importance to perception with higher quality. Face region in the input stream is allocated an above-average bit-count to yield a better visual quality than the background regions. It is the aim of this work to develop and extensively evaluate the strategy of uneven bit-allocation and also to identify its limitations.




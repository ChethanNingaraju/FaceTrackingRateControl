\chapter{Introduction}
\pagenumbering{arabic}%Ab hier, werden arabische Zahlen benutzt
\setcounter{page}{1}%Mit Abschnitt 1 beginnt die Seitennummerierung neu.
\thispagestyle{empty}

%Die Einleitung soll zum eigentlichen Themengebiet hinf�hren und die
%Motivation f�r die Arbeit liefern. Am Schlu� der Einleitung wird
%weiterhin noch eine �bersicht �ber die restliche Arbeit gegeben.
In recent years, there is an increasing demand for high-quality video conferencing solutions. Due to the availability of high-speed Internet, video conferencing has proved to be an efficient alternative to face-to-face meetings. Video telephony has grown into a multi-billion dollar industry and has a huge commercial significance. To address this growing need there has been a constant improvement in low-delay video coding techniques in addition to better techniques to ensure low-delay transmission reliability at the network level. The tremendous increase in smartphone usage has led to an increase in video telephony over cellular networks whose bandwidth is highly constrained. Therefore, it is very important to develop methods of delivering high quality video with less bandwidth requirement. 

The most commonly used video coding standards like H.264/AVC have been designed to exploit the spatial and temporal redundancies in the input video stream to achieve high data compression. The techniques of spatial and temporal prediction form the core principle of these video coding standards \cite{h.264-overview}. However, after encoding the video the perceptual redundancies still remain since human attention does not focus on the whole scene but only a small region of fixation called region-of-interest (ROI) \cite{Perception-model-of-face}. Therefore, reducing the perceptual redundancy gives a new dimension towards achieving lower bit-rate at acceptable perceptual quality. This work proposes a region-of-interest based bitrate control scheme for low-delay video encoding to exploit the perceptual redundancies.

In this work, the salient region of the frame which is the face of the participant in a video conference is identified. Since the attention of the viewer is mostly focused on the face of the other participants during a video conference call, improving the quality of the face region (ROI) can improve the overall perceptual quality. In this work, ideal capture conditions are assumed and the result of face tracking is used directly as supplementary information for the H.264/AVC encoder's bitrate control. This work explores the methods of ROI-based encoding to exploit the available bandwidth to encode regions that are of high importance to perception with higher quality. Face region in the input stream is allocated an above-average bit-count to yield a better visual quality than the background regions. It is the aim of this work to develop and extensively evaluate the strategy of uneven bit-allocation and also to identify its limitations.

In this work, OpenCV implementation of face detection using Haar feature-based cascade classifiers is used to detect the face and mark it as ROI. An optimization technique using motion vector-based variable interval face detection is proposed for real-time face detection in a video stream. This work proposes two different approaches of ROI-based encoding to improve the quality of the face region. The first approach is to use a negative QP offset for the macroblocks belonging to ROI. The QP offsets reduce the QP of ROI macroblocks resulting in a higher ROI quality. This approach is used to assess the effect of increasing the magnitude of the quality difference between ROI and non-ROI on the perceived visual quality. The second approach proposes a ROI-based bit-allocation scheme which allocates a higher proportion of bits to ROI considering the importance of the face region to the perceptual quality. This approach differs from other ROI-based bit-allocation schemes by considering the content properties to compute the optimal amount of bits for ROI and non-ROI regions. The spatial and temporal complexities of ROI and non-ROI are used to split the available bandwidth between ROI and non-ROI parts. The behavior of ROI-based bitrate control is compared with that of conventional bitrate control to make sure that ROI-based encoding does not alter the behavior of bitrate control in any undesirable manner.

The remainder of this thesis is organized as follows. Chapter \ref{chapter:Background} gives an overview of the hybrid video coding used in H.264, functionality of the bitrate control module and the concept of ROI based encoding. In Chapter \ref{chapter:Literature_survey}, a literature review is presented which discusses related works in the field of ROI-based encoding. An insight into limitations of the earlier works is also presented in this chapter. A detailed overview of the low-delay bitrate control module \cite{JVTF086} used in this work is presented in chapter \ref{chapter:used-bitrate-control-overview}. Chapter \ref{chapter:study_setup} deals with explanation of the setup used in this work along with the assessment techniques to evaluate ROI-based encoding approaches. The procedure for marking ROI in a frame using face detection and the proposed optimization technique for real-time face detection in a video stream is discussed in \Cref{chapter:FaceDetection}.The proposed bitrate control for ROI-based encoding is presented in Chapter \ref{chapter:ROI-RC}. The conclusion for this thesis work along with some future directions are discussed in chapter \ref{chapter:conclusion}.



